\documentclass{article}
\usepackage[noindent]{ctexcap}
\usepackage{amsmath,amssymb,amsthm,pifont,curves} % math package
\usepackage{multicol} % multicolumns
\usepackage{color}

\title{Active Portfolio Management}
\author{Gong Yifan}
\date{\today}

\begin{document}
\maketitle
\tableofcontents

\section{Chapter 4}
1.期望收益可以分解为:无风险收益/时间溢价、基准上的暴露$\beta_n\dot\mu_B$、超额基准收益上的暴露$\beta_n\dot\Delta{f_B}$、alpha.\\
说明:基准上的暴露中,$\mu_B$指的是基准在非常长期(70年+)上的平均收益,对股票市场来说,3\%~7\%一般是合理的。而超额基准收益,指的是短期内,
基准偏离长期收益的收益$\Delta{f_B}.$\\
2.考虑均值方差效用$U[P] = f_P - \lambda_T\dot\sigma^2_P$, 其中$\lambda_T$的选取,可以选择$$\lambda_T = \frac{\mu_B}{2\sigma^2_B}$$. 
在该参数下,考虑由基准和现金构成的组合,最大化效用,得到的是资本市场线和有效前沿的交点。


\end{document}
