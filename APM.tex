\documentclass{article}
\usepackage[noindent]{ctexcap}
\usepackage{amsmath,amssymb,amsthm,pifont,curves} % math package
\usepackage{multicol} % multicolumns
\usepackage{color}

\title{Active Portfolio Management}
\author{Gong Yifan}
\date{\today}

\begin{document}
\maketitle
\tableofcontents

\section{Chapter 4}
1.期望收益可以分解为:无风险收益/时间溢价、基准上的暴露$\beta_n\mu_B$、超额基准收益上的暴露$\beta_n\Delta{f_B}$、alpha.\\
说明:基准上的暴露中,$\mu_B$指的是基准在非常长期(70年+)上的平均收益,对股票市场来说,3\%-7\%一般是合理的。而超额基准收益,指的是短期内,
基准偏离长期收益的收益$\Delta{f_B}.$\\
2.考虑均值方差效用$U[P] = f_P - \lambda_T\sigma^2_P$, 其中$\lambda_T$的选取,可以选择$$\lambda_T = \frac{\mu_B}{2\sigma^2_B}$$. 
在该参数下,考虑由基准和现金构成的组合,最大化效用,得到的是资本市场线和有效前沿的交点。\\
3.超额收益来源于基准择时(获取超额基准收益)或者选股(获取alpha)\\
4.残差收益$\theta_P = r_P - \beta_P r_B$,主动收益$r_{PA} = r_P - r_B = \theta_P + \beta_{PA} r_B$

\section{Chapter 5}
1.事前看,alpha是对残差收益的预测。事后看,alpha是已实现的残差收益的平均。\\
对收益进行回归有\begin{align}
    r_P(t) = \alpha_P+\beta_P\dots r_B(t) + \epsilon_P(t)
\end{align}
那么策略的残差收益为$\theta_P(t) = \alpha_P+\epsilon_P(t).$其中$\alpha_P$是平均残差收益,$\epsilon_P(t)$是均值为0的随机项。\\
2.$IR_P = \frac{\alpha_P}{\omega_P}$\\
3.IR和基金经理的激进程度无关(收益和标准差波动尺度相同),但和计算时间区间有关(收益与t正相关,标准差与根号t正相关)

\section{Chapter 6}
1.fundamental law:$IR = IC*\sqrt{BR}.$~BR是因子数量/相互独立的主动决策的次数(i.e. 成长股和高利润股不是独立的)。

\section{Chapter 10}
1.预测的层次:naive, raw, refined forecast\\
naive forecast是共识的预期收益,即benchmark,是无信息的预测。\\
raw forecast是基于未处理的原始信息做出的预测。\\
refined forecast是通过预测公式,对raw forecast进行转化后得到的。公式为
\begin{align}
    E\{r|g\} = E\{r\}+Cov\{r,g\}\cdot Var^{-1}\{g\}\cdot (g-E\{g\})
\end{align}
其中~~~~\begin{minipage}{0.7\linewidth}
    r = 超额收益向量(N个资产)\\
    g = raw forecast(K个forecast)\\
    E\{r\} = naive forecast\\
    E\{g\} = expected forecast\\
    E\{r|g\} = informed expected forecast
\end{minipage}\\
\\
定义refined forecast为通过观察g得到的expected return的变化,即
\begin{align}
\Phi = E\{r|g\} - E\{r\} = Cov\{r,g\}\cdot Var^{-1}\{g\}\cdot (g-E\{g\})
\end{align}
由$Cov\{r,g\} = Corr\{r,g\}\cdot Std\{r\}\cdot Std\{g\},$\\
\begin{align}
\Phi = Std\{r\}\cdot Corr\{r,g\}\cdot \left(\frac{g-E\{g\}}{Std\{g\}}\right)
\end{align}
上式即为预测公式:refined forecast = volatility·IC·score
\end{document}
